\section{Metoder og planlægning}
I løbet af projektet har gruppen benyttet både Unified Process (UP) og Scrum. Overordnet set har gruppen fulgt UPs faser og iterationer fra inceptionsfasen til transitionsfasen, dette har givet struktur og målrettet det forberedende arbejde op til elaborationsfasen, hvor det faktiske udviklingsarbejde begyndte. Det var i denne elaborationsfase vi gjorde brug af Scrums agile projektudviklings værktøj. 
Der vil ikke blive yderligere redegjort for arbejdet i inceptionsfasen, da det alt sammen er dokumenteret i inceptionsdokumentet i bilag G.  \\
Brugen af Scrum i elaborationsfasen og tilmed konstruktionsfasen, har fungeret godt for gruppen. Dette skyldes at hvert Scrum sprint har givet et "releaseable" produkt, lige vel som hver iteration i UP kræver. Ydermere har Scrums sprint metode været, en god hjælp til at fastlæge deadlines, for enkelte dele af udviklingsarbejdet.

\subsection{Plan i elaborationsfasen}
Gruppen fik i inceptionsfasen udarbejdet en række krav og brugsmønstre, som kan ses i afsnit 4.1 tabel \ref{tab:FuncMoscow}, disse lå til grund for arbejdet med Scrum. Til første sprint startede gruppen ud med at oprette en product backlog, ud fra de højest prioriteret krav og brugsmøsntre. Til strukturering af produkt backloggen har gruppen benyttet sig af GitHub projects. Backloggen har gruppen benyttet ved at der i fællesskab, blev tilføjet en række opgaver til en To Do liste, i starten af hvert sprint, hvorpå medlemmerne kan påtage sig en eller flere opgave. Når en opgave er tildelt, markeres det at den er under udvikling, hvorefter den ved færdiggørelse rykkes til gennemsyn og til sidst står som afsluttet. \\
Til hver iteration har gruppen gennemgået 2 sprints, som i tilsammen løber op i 4 sprints. 
\begin{itemize}
    \item \textbf{Sprint 1} \\
    I første sprint delte gruppen sig op i 3 par, hvor hvert par fik tildelt en række opgaver. Første iteration bestod hovedsageligt i at få opbygget skelettet af programmet og et udkast til GUI'en.
    \item \textbf{Sprint 2} \\
    I andet sprint delte gruppen sig op i 2, hvorpå en gruppe stod for implementering af at skrive til og læse fra tekst filer. Den anden gruppe stod for at tilføje funktionalitet til at modtage input og vise output i til brugeren fra GUI'en.
    \item \textbf{Sprint 3}
    I sprint 3 gik projektet fra 1. iteration til 2. iteration, hvor persistenslaget blev skiftet ud fra tekst filer til en relationel SQL database. 
\end{itemize}