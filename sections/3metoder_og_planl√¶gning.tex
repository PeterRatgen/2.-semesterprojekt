\section{Metoder og planlægning}
I løbet af projektet har gruppen benyttet både Unified Process (UP) og Scrum. Overordnet set har gruppen fulgt UPs faser og iterationer fra inceptionsfasen til transitionsfasen, dette har givet struktur og målrettet det forberedende arbejde op til elaborationsfasen, hvor det faktiske udviklingsarbejde begyndte. Det var i denne elaborationsfase vi gjorde brug af Scrums agile projektudviklings værktøj. 
Der vil ikke blive yderligere redegjort for arbejdet i inceptionsfasen, da det alt sammen er dokumenteret i inceptionsdokumentet i bilag G.  \\
Brugen af Scrum i elaborationsfasen og tilmed konstruktionsfasen, har fungeret godt for gruppen. Dette skyldes at hvert Scrum sprint har givet et "releaseable" produkt, lige vel som hver iteration i UP kræver. Ydermere har Scrums sprint metode været, en god hjælp til at fastlæge deadlines, for enkelte dele af udviklingsarbejdet.

\subsection{Plan i elaborationsfasen}
Gruppen fik i inceptionsfasen udarbejdet en række krav og brugsmønstre, som kan ses i afsnit 4.1 tabel \ref{tab:FuncMoscow}, disse lå til grund for arbejdet med Scrum. Til første sprint startede gruppen ud med at oprette en product backlog, ud fra de højest prioriteret krav og brugsmøsntre. 