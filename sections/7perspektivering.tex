\section{Perspektivering}
%Er den fundne løsning brugbar i anden sammenhæng?Hvad  bidrager  løsningen  og  den  opn ̊aede  viden  til. 
%Fremtidigt  arbejde  (næste  skridt  i  projektet,  hvis  I havde mere tid).

Dette projekt har til formål at udvikle et system til håndtering af krediteringer for TV 2, og erstatte de traditionelle rulletekster. Systemet er udviklet på baggrund af TV 2s case beskrivelse og et opfølgende interview med TV 2s udvikler, Morten Lehm. Derfor er systemet udviklet målrettet efter deres ønsker og krav. \\
Dette er dog ikke ensbetydende med at systemet eller dele af systemet ikke er brugbar i anden sammenhæng. Morten Lehm kunne fortælle under interviewet, at et sådan system har været på tale i branchen og der er interesse fra flere parter. Bl.a. kunne han fortælle at Danmark Radio også kunne være interesseret i en sådan løsning til visning af krediteringer. Udover Danmarks Radio har Tv-producenter også vist interesse for systemet, da de traditionelle rulletekster til tider kan være ulæselig, pga. for mange krediteringer i forhold til det korte tidsinterval. Der er altså interesse for system fra andre en TV 2. Måden hvorpå programmet er bygget op, med 3-lags modellen, giver stor mulighed for at benytte det i andre sammenhænge. Som denne udgave af programmet fremstår for brugen, er med tanken om at det er et krediteringssystem for TV2, hvorfor det fremstår med TV 2s logo og farver. Dette kommer dog kun til udtryk i præsentationslaget, hvilket vil sige at, hvis f.eks. Danmarks Radio ønsker en udgave af programmet med den samme funktionalitet, kræver det ikke nogen betydelig indgreb i de øvrige lag. \\
Udover at 3-lag arkitekturen giver mulighed for udskiftning af præsentationlaget, giver det lige vel mulighed for at udskifte den nuværende SQL database i persistenslaget, med minimalt indgreb i de øvrige lag. \\\\

\textbf{Det fremtidige arbejde} \\
Skulle gruppen forstætte med en 3. iteration af projektet står det klart hvad de næste skridt er. Da gruppen valgte at prioritere en omstrukturering af store dele af systemet i starten af 2. iteration, for at opnå en bedre lagdelt struktur, er ikke alle de ønskede krav opnået. Det disse krav der vil stå først på listen i det fremtidige arbejde, og her er der tale om:
\begin{itemize}
    \item Adgangskontrol (F10)\\
    Adgangskontrollen er kun implementeret i form af knapper, hvor brugeren vælger hvilken rolle der ønskes at være logget ind som, om det producer, moderator eller administrator. Her skal et login system implementeres, som administreres af system administratoren, og dermed også giver denne yderligere rettigheder. 
    \item Anvendelse af Data (F06) \\
    TV 2 forspurgte en funktion, til at eksportere en specifik mængde af data til formatter som f.eks. XML og CSV, så de nemt kan anvende data på anden vis.
    \item Firmaer og grupper \\
    Det bør være muligt at krediterer firmaer og grupper til produktioner, det har dog ikke været prioriteret. 
\end{itemize}