\subsection{Detaljeret krav}

Ud fra de overordnede krav er enkelte krav/brugsmøsntre udvalgt til en mere detaljeret beskrivelse. Brugsmønstrene er udvalgt på baggrund af vigtigheden for projektet, derfor er det brugsmøsntrene "registrering af kreditering" og "søgning på kreditering", der er medtaget i rapporten. 

\begin{longtable}{|p{150mm}|}
\hline
\textbf{Brugsmønster:} Registrering af kreditering.           \\ \hline
\textbf{ID:} B01                                     \\ \hline
\textbf{Primære aktører:} Producenter og firmaer (Opretter)       \\ \hline
\textbf{Sekundære aktører:}      \\ \hline
\textbf{Kort beskrivelse:} Producent for showet eller en SoMe-ansvarlig fra et firma kan logge ind på systemet og oprette showet, ved at udfylde en skabelon.          \\ \hline
\textbf{Præconditioner} Man skal være logget ind som enten producent eller firma            \\ \hline
\textbf{Hovedhændelsesforløb}
    \begin{enumerate}
        \setlength{\itemsep}{0pt}
        \item Producent eller firma logger ind på systemet.
        \item Opretteren navigerer til siden med oprettelse af show.
        \item Opretteren udfylder skabelon til oprettelse af show som indbærer:
        \begin{itemize}
            \setlength{\itemsep}{0pt}
            \item Navn på show og evt. serie ID.
            \item Dato for første udsendelse.
            \item Navn på producent.
            \item kategori.
            \item mm.
        \end{itemize}
        \item Krediteret personer tilføjes til show.
        \begin{itemize}
            \setlength{\itemsep}{0pt}
            \item Opretteren skal kunne hente personer, som eksisterer i databasen og tilføje dem til showet.
            \item Ved tilføjelse af ikke eksisterende person: se alternativt hændelsesforløb.
        \end{itemize}
        \item Den udfyldte skabelon registreres og sendes til databasen som "ikke godkendt", hvorefter en TV 2 moderator kan tilgå krediteringen og godkende den.
    \end{enumerate}  \\ \hline
\textbf{Postkonditioner:} Showet står som "ikke godkendt" indtil en ansat fra TV 2 har kvalitetstjekket indholdet og godkendt det. \\\hline
\textbf{Alternative hændelsesforløb:} Skridt 4. Ved tilføjelse af personer, der ikke allerede eksisterer i databasen, får opretteren mulighed for tilføjelse af den nye person ved et pop-up vindue, for yderligere specificering se B02 - Oprettelse af person. \\\hline
    \caption{Detaljeret brugsmønster over B03}
    \label{tab:Detaljeret_brugsmønsterdiagram_B03}
\end{longtable}

\begin{longtable}{|p{150mm}|}
\hline
\textbf{Brugsmønster:} Søgning på kreditering\\ \hline
\textbf{ID:} B03         \\ \hline
\textbf{Primære aktører:} Bruger          \\ \hline
\textbf{Sekundære aktører:}        \\ \hline
\textbf{Kort beskrivelse:} Brugeren bliver mødt af et søgefelt, hvor de indtaster deres ønskede søgekriterier og vælger herefter hvilken kreditering, de ønsker at få vist         \\ \hline
\textbf{Præconditioner:} Brugeren har et søgekriterie i form af navn på show/person eller ID        \\ \hline
\textbf{Hovedhændelsesforløb:}
    \begin{enumerate}
    \setlength{\itemsep}{0pt}
        \item Brugeren bliver mødt af et tomt søgefelt
        \item Brugeren indtaster et søgekriterie i form af navn på show/person eller ID
        \item Brugeren søger med de givne søgekriterier
        \item Brugeren får vist en liste af krediteringer, der matcher søgekriterierne
        \item Brugeren vælger her den ønskede kreditering 
        \item Brugeren får nu vist den specifikke kreditering som de har valgt
    \end{enumerate}
\\ \hline
\textbf{Postkonditioner:} Logiklaget har leveret informationer til brugeren, og brugeren har adgang til de krediteringsinformationer de ønsker         \\ \hline
\textbf{Alternative hændelsesforløb:}
\begin{itemize}
    \item Skridt 5: Hvis brugeren ikke kan finde den ønskede kreditering kan de fortage endnu en søgning eller opgive
\end{itemize}\\ \hline
    \caption{Detaljeret brugsmønster over B01}
    \label{tab:Detaljeret_brugsmønsterdiagram_B01}

\end{longtable}

\subsubsection{Supplerende krav}
En detaljeret beskrivelse af de supplerende krav kan med fordel stilles op ved hjælp Furps+ model. Med Furps+ kategoriseres kravene efter hvilke område de vedrører. 



\begin{itemize}
    \item \textbf{Reliability} \newline
    Under dette punkt placeres de krav, som bliver sat til pålideligheden af programmet. I vores tilfælde har vi 2 krav, som omhandler oppetid og sikkerhed for programmet. 
        \begin{table}[H]
        \centering
        \begin{tabular}{|p{30mm}|p{90mm}|}
        \hline
            Oppetid &  TV 2 har et krav om at systemet har en oppetid på 99,7 - 99,8\% .
        \\ \hline
            Sikkerhed & TV 2 stiller et meget stærkt krav om, at der er styr på, at al data i systemet er korrekt, da data skal anvendes for at sikre, at krediterede personer får den betaling, de har ret til.
        \\ \hline
        \end{tabular}
            \caption{Supplerende krav - Reliability}
            \label{tab:Reliablity}
        \end{table}
    
    \item \textbf{Performance} \newline
    Til dette punkt har vi et enkelt krav, som ikke er yderligere specificeret end, at programmet skal have en svartid der er så lav som muligt.
        \begin{table}[H]
            \centering
            \begin{tabular}{|p{30mm}|p{90mm}|}
            \hline
                Svartid & TV 2 Ønsker at svartiden på systemet skal være så lav som muligt.
            \\ \hline
            \end{tabular}
                \caption{Supplerende krav - Performance}
                \label{tab:Performance}
        \end{table}


    \item \textbf{Plus} \newline
    \textbf{Design constraints:} Her er det vi placere de specifikke begrænsning vores system er pålagt. Kravet om 3-lags arkitektur placeres her da det er et specifikt krav til systemet stillet af SDU.
        \begin{table}[H]
        \centering
            \begin{tabular}{|p{30mm}|p{90mm}|}
            \hline
                3-lags arkitektur & Det ønskes at programmet følger en 3-lags arkitektur.
            \\ \hline
            \end{tabular}
        \caption{Supplerende krav - Design constraints}
        \label{tab:Design_constraints}
        \end{table}
        
    \textbf{Implementation requirements:}
    Implementerings krav stiller retningslinjer for programmørenes implementering, og de blot skal forholde sig til standarderne eller noget specifikt. Her har vi et krav stillet TV 2, som omhandler måden hvorpå en person registreres i systemet. Det skal her være muligt at differentiere mellem personer med identiske navne. Dette vurdere vi til at være delvist opnået, da det er muligt at oprette personer med identiske navne, og de vil få tildelt unikke ID'er. Dette er dog ikke tydliggjordt for brugeren, hvorfor vi mener den er delvist opnået. 
    
    
        \begin{table}[H]
        \centering
        \begin{tabular}{|p{30mm}|p{90mm}|}
        \hline
            Unikke personer & Der er et krav om, at man skal kunne differentiere personer med f.eks. identiske navne.
        \\ \hline
        \end{tabular}
            \caption{Supplerende krav - Implementation requirements}
            \label{tab:Implementation_requirements}
        \end{table}
        
    \textbf{Interface requirements:}
    Interface requrements kigger på om der nogle interfaces, som programmet skal kunne på eller med. Til dette projekt har TV 2 haft et ønske om at programmet skal interagere med deres øvrige programmer, dette har dog ikke været muligt at udfører i denne udgave af produktet.
        \begin{table}[H]
        \centering
        \begin{tabular}{|p{30mm}|p{90mm}|}
        \hline
            Integration med andre systemer & TV 2 har et ønske om at programmet skal kunne integreres med en række af de systemer, de anvender i forvejen. Eksempler kunne være EPG og login-systemet til TV 2 Play.
        \\ \hline
        \end{tabular}
            \caption{Supplerende krav - Interface requirements}
            \label{tab:Interface_requirements}
        \end{table}
    
\end{itemize}
