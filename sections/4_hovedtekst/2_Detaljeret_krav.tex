\subsection{Detaljeret krav}

Ud fra de overordnede krav er enkelte krav/brugsmøsntre udvalgt til en mere detaljeret beskrivelse. Brugsmønstrene er udvalgt på baggrund af vigtigheden for projektet, derfor er det brugsmøsntrene "registrering af kreditering" og "søgning på kreditering", der er medtaget i rapporten. 

\begin{longtable}{|p{150mm}|}
\hline
\textbf{Brugsmønster:} Registrering af kreditering.           \\ \hline
\textbf{ID:} B01                                     \\ \hline
\textbf{Primære aktører:} Producenter og firmaer (Opretter)       \\ \hline
\textbf{Sekundære aktører:}      \\ \hline
\textbf{Kort beskrivelse:} Producent for showet eller en SoMe-ansvarlig fra et firma kan logge ind på systemet og oprette showet, ved at udfylde en skabelon.          \\ \hline
\textbf{Præconditioner} Man skal være logget ind som enten producent eller firma            \\ \hline
\textbf{Hovedhændelsesforløb}
    \begin{enumerate}
        \setlength{\itemsep}{0pt}
        \item Producent eller firma logger ind på systemet.
        \item Opretteren navigerer til siden med oprettelse af show.
        \item Opretteren udfylder skabelon til oprettelse af show som indbærer:
        \begin{itemize}
            \setlength{\itemsep}{0pt}
            \item Navn på show og evt. serie ID.
            \item Dato for første udsendelse.
            \item Navn på producent.
            \item kategori.
            \item mm.
        \end{itemize}
        \item Krediteret personer tilføjes til show.
        \begin{itemize}
            \setlength{\itemsep}{0pt}
            \item Opretteren skal kunne hente personer, som eksisterer i databasen og tilføje dem til showet.
            \item Ved tilføjelse af ikke eksisterende person: se alternativt hændelsesforløb.
        \end{itemize}
        \item Den udfyldte skabelon registreres og sendes til databasen som "ikke godkendt", hvorefter en TV 2 moderator kan tilgå krediteringen og godkende den.
    \end{enumerate}  \\ \hline
\textbf{Postkonditioner:} Showet står som "ikke godkendt" indtil en ansat fra TV 2 har kvalitetstjekket indholdet og godkendt det. \\\hline
\textbf{Alternative hændelsesforløb:} Skridt 4. Ved tilføjelse af personer, der ikke allerede eksisterer i databasen, får opretteren mulighed for tilføjelse af den nye person ved et pop-up vindue, for yderligere specificering se B02 - Oprettelse af person. \\\hline
    \caption{Detaljeret brugsmønster over B03}
    \label{tab:Detaljeret_brugsmønsterdiagram_B03}
\end{longtable}

\begin{longtable}{|p{150mm}|}
\hline
\textbf{Brugsmønster:} Søgning på kreditering\\ \hline
\textbf{ID:} B03         \\ \hline
\textbf{Primære aktører:} Bruger          \\ \hline
\textbf{Sekundære aktører:}        \\ \hline
\textbf{Kort beskrivelse:} Brugeren bliver mødt af et søgefelt, hvor de indtaster deres ønskede søgekriterier og vælger herefter hvilken kreditering, de ønsker at få vist         \\ \hline
\textbf{Præconditioner:} Brugeren har et søgekriterie i form af navn på show/person eller ID        \\ \hline
\textbf{Hovedhændelsesforløb:}
    \begin{enumerate}
    \setlength{\itemsep}{0pt}
        \item Brugeren bliver mødt af et tomt søgefelt
        \item Brugeren indtaster et søgekriterie i form af navn på show/person eller ID
        \item Brugeren søger med de givne søgekriterier
        \item Brugeren får vist en liste af krediteringer, der matcher søgekriterierne
        \item Brugeren vælger her den ønskede kreditering 
        \item Brugeren får nu vist den specifikke kreditering som de har valgt
    \end{enumerate}
\\ \hline
\textbf{Postkonditioner:} Logiklaget har leveret informationer til brugeren, og brugeren har adgang til de krediteringsinformationer de ønsker         \\ \hline
\textbf{Alternative hændelsesforløb:}
\begin{itemize}
    \item Skridt 5: Hvis brugeren ikke kan finde den ønskede kreditering kan de fortage endnu en søgning eller opgive
\end{itemize}\\ \hline
    \caption{Detaljeret brugsmønster over B01}
    \label{tab:Detaljeret_brugsmønsterdiagram_B01}

\end{longtable}

