\newpage
\subsection{Implementation}

Formålet med dette afsnit er at gå fra vores design til kode.

\subsubsection{Søgning i krediteringer}%
\label{ssub:sogning_i_krediteringer}

Søgninger i krediteringer baserer sig på brugsmønster B03. Operationskontrakten
for dette brugsmønster ses i tabel \ref{tab:OperationsKontraktB03}. Det
udspecificerede sekvensdiagram ses på figur \ref{fig:B03OSDDesign}.

Fra \texttt{Menu.fxml} kaldes \texttt{submitSearch()} i controlleren
\texttt{MenuController}, ved tryk på søgeknappen. Hvis man kaster et blik på
MVC-modellen set på figur \ref{fig:mvc}, kan man at der her kaldes fra viewet i
\texttt{.fxml} filen til controlleren. \texttt{MenuController} sætter derpå et
nyt view, \texttt{SearchResult.fxml}. Dette view skal vise resultaterne af
søgningen. Dette ses i nedenstående snippet. Der sættes også en ny controller,
til at administrere det nye view.

\begin{lstlisting}
setContentPane("SearchResult.fxml", (Object) SearchController.getInstance());
SearchController.getInstance().setContent();
\end{lstlisting}

Derpå kalder \texttt{SearchController},

\begin{lstlisting}
SearchList.setItems(ApplicationManager.getInstance().search(searchString));
\end{lstlisting}

hvorpå at \texttt{ApplicationManager} i domænelaget står for at facilitere
søgningen, ved at kalde de korrekte Managers for at lave søgningen. Her bliver
\texttt{PersonManager} kaldt, hvorpå 
