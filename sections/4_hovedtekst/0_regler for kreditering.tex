\subsection{Regler for kreditering}

%Hvordan er reglerne for krediteringer for danskproducerede programmer?•

TV 2 er underlagt en række af bestemmelser for visning af krediteringer, som alt sammen skal overholdes, med en maksimal varighed på 30 sekunder, når det vises på traditionel vis med rulletekster. Selve krediteringerne har til formål at \textit{kreditere de personer, som ophavsretligt, overenskomstmæssigt eller kontraktligt har krav på det, samt at orientere seerne om, hvem der har bidraget selvstændigt og kreativt til et programs tilblivelse. Efter ophavsretsloven har ophavsmanden krav på at blive krediteret i overensstemmelse med god skik, og denne ret kan ikke tilsidesættes ved aftale.}\cite{url_kredit_regler} \\
Hvilket er de regler og overensstemmelser programmet skal opfylde og overholde, for at kunne erstatte de traditionelle rulletekster. Derudover er det vigtigt at vide hvilke informationer der skal og må inkluderes i krediteringerne. Her skal reglerne omkring GDPR først og fremmest overholdes, da de indtastede oplysninger vil blive frit tilgængeligt. For at krediteringerne giver mening, skal navnet på den krediteret person som minimum inkluderes, for at krediteringen giver mening. 

%Hvilke informationer skal og m ̊a inkluderes i krediteringer?
%Hvordan kan s ̊adan et program gøres mest mulig brugervenligt(tilgængeligt) for b ̊adeseer og administrator?
