\clearpage
\section{Resumé}
\begin{comment}
En  kort  introduktion  til  projektet  -  hvad  blev der arbejdet med og hvorfor.

•Problemformuleringen og vigtige afgrænsninger.

•Metode - hvordan angreb I problemet og hvordan realiserede I løsningen (hvem, hvad, hvorn ̊ar og hvorfor

)•Hovedresultater og konklusioner – hvad kom derud af arbejde
\end{comment}

Denne rapport omhandler udviklingsarbejdet, inkluderende arbejdsprocesser, beslutninger og resultater for projektet af gruppe SE04 på 2. semester på software engineering. Projektet bygger på en problematik TV 2 står overfor, med hensyn til visning af kreditering. Det foregår på nuværende tidspunkt med traditionelle rulletekster efter endt program. Dette er begrænset til maksimalt at må vare 30 sekunder, hvilket skaber problematikken. 30 sekunder er ikke altid nok til at vise alle krediteringer, hvilket fører til brud på ophavsret og andre aftaler. Ydermere vil frigørelsen af de 30 sekunder kunne øge TV 2s årlige indkomst med op imod 60 millioner. Dette skaber grundlaget for projektet, der skal være et digitaliseret system til håndtering og visning af kreditering, som skal kunne erstatte rulleteksterne og derved løse problematikken. \\

Til opgave har TV 2 udleveret en case beskrivelse som gruppen har analyseret og kommet frem til følgende probelemformulering med tilhørende underspørgsmål, samt afgrænsninger:\\

\textbf{Hvordan kan vi udvikle en prototype til et krediteringssystem, der vil kunne erstatte de klassiske rulletekster efter et afsluttet program?}
\begin{itemize}
    \item Hvordan er reglerne for krediteringer for danskproducerede programmer?
    \item Hvilke informationer skal og må inkluderes i krediteringer?
    \item Hvordan kan sådan et program gøres mest mulig brugervenligt for både seer og administrator?
    \item Hvordan kan vi sikre et velstruktureret program, der mindsker fejl og ventetid, samt strukturere databasen således, at der undgås fejl, duplikering af værdier og null-værdier?
\end{itemize}
Projektet er afgrænset til udelukkede at håndtere krediteringer for dansk produceret indhold. Derudover er det målrettet til TV 2 og deres ønsker, dog bliver ikke alle krav opfyldt, og programmet er derfor kun en prototype og ikke et færdigt produkt der kan erstatte rulleteksterne. \\
Projektet arbejds- og udviklingsproces har fulgt de to modeller Unified Process (UP) og Scrum. Ved at følge UP har gruppen i den første fase af projektet udarbejdet et inceptionsdokumtet, hvorpå forudsætningerne for projektet blev analyseret, og det blev der konkluderet at projektet er værd at arbejde videre på. Arbejdet i inceptionsfasen dannede yderligere grundlagt for udviklingsarbejdet i den efterfølgende fase. \\
Underuviklingsarbejdet har gruppen opnået en række af de målsatte krav, og formået at bygge systemet op med en velstruktureret lagdelt arkitektur. Dog er systemet ikke fuldendt og er kun en prototype til et endeligt program der skal erstatte de traditionelle rulletekster.
