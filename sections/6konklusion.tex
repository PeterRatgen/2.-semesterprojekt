\section{Konklusion}
Projektets formål lød på at udvikle en prototype til et program, til at erstatte rulletekster efter et tv-show, for TV 2.
Dette projektforløb er blevet udarbejdet ved benyttelse af Unified Process og SCRUM. Projektet er forløbet under flere iterationer. For hver iteration undergik gruppen 2, i alt 4 sprints. Denne udarbejdningsstil gør det muligt for gruppen vende tilbage til starten efter en iteration, og forbedre de ting som er blevet lært. Derved udforme det bedst mulige resultat.

Der er mange problemer når det kommer til at håndtere krediteringer, specifikt for TV 2 skal regler som Ophavsretsloven, TV 2s overenskomster, Kontrakter for ophavsmænd og udøvende kunstnere og almindelig god skik overholdes. Yderligere skal krediteringer vises for de personer, som ophavsretligt, overenskomst eller kontraktligt har krav på det, samt at orientere seerne om, hvem der har bidraget selvstændigt og kreativt til et programs tilblivelse.

For at sikre et velstruktureret program, har gruppen gjort brug af et 3-lags arkitektur, for at mindske uoverskuelighed i et stort projekt, samt sikre mange forespørgsler frem og tilbage mellem de enkelte artifakter i systemet. Yderligere er der gjort brug af strukturelle desgin mønstre, som facader. disse facader hjælper programmet både med kompleksiteten og uønskede afhængigheder. Hertil er der gjort brug af singletons og factory-metoder for at mindske kompleksitet.

I løbet af projektet har gruppen udarbejdet detaljeret brugsmønstre, disse giver indblik i hvordan gruppen bedst muligt kan komme frem til at implementere et produkt, som har fokus på brugervenlighed. For at gøre det mest brugervenligt for seer, implementeres programmets funktioner således at brugeren ikke behøver at have en konto eller logge ind. Altså har brugeren fuld adgang til produktets funktioner, som de skal bruge. Yderligere implementeres en søgefunktion, for at gøre det nemmest muligt for seeren at søge på hvad de ville have. Produktet skal også have ekstra muligheder for kunne redigere, slette, oprette af forskellige krediteringer. Det ville være besværligt hvis der skulle et helt andet produkt til for at dette kunne gøres. Derfor implementeres alle funktionerne ind i produktet. Funktionerne gemmes for de brugere som ikke skal have adgang til dem. 

Igennem dette projektforløb har gruppen fundet frem til at der er mange forskellige regler for krediteringer. samt er der rigtig mange informationer der skal tages med og kan tages med i kreditering. Der er mange metoder hvorpå et velstruktureret program kan udvikles på og dette produkt er hvordan denne gruppe har dannet en løsning. 
På baggrund af dette kan gruppen sige at det endelige produkt er den bedst mulige prototype til et krediteringssystem, gruppen har haft kunne producere.