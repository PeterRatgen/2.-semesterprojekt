\newpage
\section{Diskussion}
%Hvad er der opn ̊aet og hvad er der ikke opn ̊aet i po-jektet i forhold til det
%forventede som beskrevet i ind-ledningen. Hvad er styrkerne og svaghederne ved
%re-sultaterne. Kunne I have opn ̊aet bedre resultater?

\subsection{Målet for projektet}%
\label{sub:malet_for_projektet}

Målet for projektet var at udvikle en prototype til et program, der skal
erstatte de klasse rulletekster efter et tv-show, for TV 2. TV 2 udformet en
case beskrivelse med en række af krav, som er blevet udvidet og prioriteret af
gruppen i første fase af projektet. Kravene blev prioriteret efter MoSCoW
modellen, hvorpå en række af kravene blev kvalificeret som et "Must" krav.
"Must" kravene er de nødvendige krav, som skal opfyldes for at vi mener
produktet har substans. Disse krav er alle blevet opfyldt og fungere efter
hensigten. Dernæst har vi en rækken af krav under "Should", som ikke har været i
højeste prioritet. En tilfredsstillende del af disse krav er opfyldt eller
delvist opfyldt.

\subsection{Programmet status}%
\label{sub:programmet}

Med de opfyldt og delvist opfyldt krav, har gruppen formået at udvikle et
kørende program, som tillader en producer at tilføje og redigere produktioner i
form af film eller serier. Tilmed kan en producer oprette og registrere
personer, og tilføje dem til de produktioner de har medvirket til udviklingen
af, og tildele dem roller og evt. karakternavn. Alt dette føjer programmet til
en online database, hvorefter en medarbejder fra TV 2, har muligheden for at
kontrollere producentens indtastninger og godkende, før det bliver tilgængeligt
for den almindelige bruger. Den almindelige bruger kan søge på person og
produktioner, og finde tilhørende oplysninger, på en nem og brugervenlig måde. 

Gruppen havde forventet at opnå enkelte krav, som ikke er blevet en realitet.
Her er der tale om kravet omkring adgangskontrol, som denne udgave af prototypen
indholder et alternativ til. Hvilket også medfører at en systemadministrator
ikke har nogle betydelige funktioner, denne var tiltænkt til håndtering af de
øvrige brugers rettigheder/roller. Derudover var det ønsket at en krediteret
person skulle have mulighed for at redigere sine egne oplysninger, hvilket
heller ikke er muligt. Til slut havde gruppen et ønske om en funktion, der
tillader at hente krediteringer i forskellige formatter, som f.eks. XML og CSV. 

Selvom gruppen ikke har formået at implementere alle de ønskede funktioner, er
det endelige resultat dog stadig tilfredsstillende. Kvaliteten af de
implementerede funktioner, er gennem udviklingsarbejdet blevet prioriteret
højere end at opfylde samtlige krav af lavere kvalitet. Skulle der arbejdes
videre på systemet er det bygget op på en sådan måde, at der ikke kræves nogen
større omstrukturering for at implementere et evt. loginsystem til
adgangskontrollen.

Projektet har kørt sideløbende med den undervisning, der ligger til
grund for at gruppens viden og færdigheder, der kræves for at udarbejde
sådan projekt, vurderer gruppen at, hvis den havde haft samme viden fra projektets
start, som de har nu, ville bedre resultater kunne være opnået. Efter første
iteration var afsluttet, gennemgik projektet en større strukturel
refaktorering. Dette har kostet tid og krævet en del arbejdsindsats, som bl.a.
skyldes i at der ikke er opnået flere krav. Hvis gruppen fra start havde lagt
mere vægt på design, der i involveringen af de designmønstre der senere blev
brugt i refaktoreringen, kunne gruppen formentlig have opnået et mere
tilfredsstillende resultat med implementering af flere brugsmønstre.
Dette viser vigtigheden af en god arkitektur, samt et godt design som et
udgangspunkt for implementeringen. 


\subsection{Testing}%
\label{ssub:testing}

Hvad angår testing af systemet, skal der udvides med flere unit-tests, samt
integrationstest for at verificere at kravene der er stillet til projektet er
opnået. Dette kunne fx være test af adgangskontrolsystemet, således at en
moderator ikke har adgang til de samme rettigheder som en administrator.

Det ville også være relevant at teste på min man kan oprette alle forskellige
typer af krediteringer i databasen, for at få vished for at interaktionen med
databasen er på plads.
