\section{Diskussion}
%Hvad er der opn ̊aet og hvad er der ikke opn ̊aet i po-jektet i forhold til det forventede som beskrevet i ind-ledningen. Hvad er styrkerne og svaghederne ved re-sultaterne. Kunne I have opn ̊aet bedre resultater?

Målet for projektet var at udvikle en prototype til et program, der skal erstatte de klasse rulletekster efter et tv-show, for TV 2. TV 2 udformet en case beskrivelse med en række af krav, som er blevet udvidet og prioriteret af gruppen i første fase af projektet. Kravene blev prioriteret efter MoSCoW modellen, hvorpå en række af kravene blev kvalificeret som et "Must" krav. "Must" kravene er de nødvendige krav, som skal opfyldes for at vi mener produktet har substans. Disse krav er alle blevet opfyldt og fungere efter hensigten. Dernæst har vi en rækken af krav under "Should", som ikke har været i højeste prioritet. En tilfredsstillende del af disse krav er opfyldt eller delvist opfyldt. \\
Med de opfyldt og delvist opfyldt krav, har gruppen formået at udvikle en kørende program, som tillader en producer at tilføje og redigere produktioner i form af film eller serier. Til med kan en producer oprette og registrer personer, og tilføje dem til de produktioner de har medvirket til udviklingen af, og tildele dem roller og evt. karakternavn. Alt dette føjer programmet til en online database, hvorefter en medarbejder fra TV 2, har muligheden for at kontrollere producentens indtastninger og godkende, før det bliver tilgængeligt for den almindelige bruger. Den almindelige bruger kan søge på person og produktioner, og finde tilhørende oplysninger, på en nem og brugervenlig måde. \\
Gruppen havde forventet at opnå enkelte krav, som ikke er blevet en realitet. Her er der tale om kravet omkring adgangskontrol, som denne udgave af prototypen indholder et alternativ til. Hvilket også medfører at en systemadministrator ikke har nogle betydelige funktioner, denne var tiltænkt til håndtering af de øvrige brugers rettigheder/roller. Derudover var det ønsket at en krediteret person skulle have mulighed for at redigere sine egne oplysninger, hvilket heller ikke er muligt. Til slut havde gruppen et ønske om en funktion, der tillader at hente krediteringer i forskellige formatter, som f.eks. XML og CSV. \\
Selvom gruppen ikke har formået at implementere alle de ønskede funktioner, er det endelige resultat dog stadig tilfredsstillende. Kvaliteten af de implementerede funktioner, er gennem udviklingsarbejdet blevet prioriteret højere end at opfylde samtlige krav af laver kvalitet. Skulle der arbejdes videre på systemet