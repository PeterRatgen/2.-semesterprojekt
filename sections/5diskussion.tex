\section{Diskussion}
%Hvad er der opn ̊aet og hvad er der ikke opn ̊aet i po-jektet i forhold til det forventede som beskrevet i ind-ledningen. Hvad er styrkerne og svaghederne ved re-sultaterne. Kunne I have opn ̊aet bedre resultater?

Målet for projektet var at udvikle en prototype til et program, der skal erstatte de klasse rulletekster efter et tv-show, for TV 2. TV 2 udformet en case beskrivelse med en række af krav, som er blevet udvidet og prioriteret af gruppen i første fase af projektet. Kravene blev prioriteret efter MoSCoW modellen, hvorpå en række af kravene blev kvalificeret som et "Must" krav. "Must" kravene er de nødvendige krav, som skal opfyldes for at vi mener produktet har substans. Disse krav er alle blevet opfyldt og fungere efter hensigten. Dernæst har vi en rækken af krav under "Should", som ikke har været i højeste prioritet. En 

