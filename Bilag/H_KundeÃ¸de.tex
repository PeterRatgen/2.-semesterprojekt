\section{Referat af kundemøde}
\label{kundemøde}
Referat af kundemøde med TV 2 udvikler Morten Lehm 

\textbf{Indeledende af Morten Lehm}
I dag er krediteringen for det meste manuelt arbejde. Der er reelt set ikke mange der ser rulleteksterne. Hvad angår at vise reklamer i stedet for de 30 sekunder man bruger på krediteringer, så er TV 2 underlagt nogle regler for hvor mange reklamer man må vise på en time. TV 2 Play samt TVTID kunne drage nytte af et nyt krediteringssystem. Noget man også kigger på er oppetid, typisk siger de 99.7 - 99.8. Svartider er også vigtigt. De kigger også på deploymenttid, hvor hurtigt skal man kunne opdatere til en ny version. Siden der er noget økonomisk inde over, er det vigtigt med noget sikkerhed. Der kunne man for eksempel lave noget logging af hvem der der ændrer noget i systemet.

\textbf{Hvem skal bruge systemet?}

Det er primært producenter der skal gøre det. Da vil TV 2s rolle være at kvalitetssikre krediteringerne.

\textbf{Hvilke konkrete ting kunne være interessant at få opsummeret?}

Det kunne være interessant at vide hvor mange forskellige producenter man har. Hvor mange man krediterer inde forskellige områder (kameramænd, skuespillere, mv.)

\textbf{Hvordan håndteres krediteringerne nu?}

Der sendes et worddokument med krediteringerne, også gemmes data i TV-systemet. Disse filer gemmes i et fileshare system.

\textbf{Hvad er god skik inden for kreditering?}

Det skal være læsbart for seeren.

\textbf{Har i undersøgt med danske film- og TV-producenter?}

De er stadigvæk interresserede, også i et få det mere automatisert, end i dag.

\textbf{Kunne en privat kunde oprette sig i systemet?}

Hvis en privat bruger skulle oprette sig i systemet, skulle det være med den bruger de bruger i forhold til TV 2-play. Dette kunen man bruge til dataopsamling og bruge data til bedre anbefalinger i TV 2 Play.

\textbf{I cases bliver det beskrevet at man kan fjerne alle rulletekster, er det inden for reglerne?}

I den oprindelige undersøgning fandt man ud af at man skulle have en reference tilservicen.

\textbf{I er interesserede i notifkationer, hvordan?}

Hvis man laver en app kunne det være på telefonen, ellers kan man lave en email. Hos dem er det primært push man bruger lige nu.

\textbf{Hvordan er rækkefølgen?}

Skuespiller er typisk de første i rækkefølgen.

\textbf{Hvem bliver brugerne, hvis folk ikke er interesserede i rulletekster?}

Personerne der er krediteret, familiemedlemmer. Så har man også et sted man kan henvise til.

\textbf{Hvad med en QR-kode som link til siden?}

QR-koden skal vises relativt længe i Picture-in-Picture.

\textbf{Unique persons?}

Hvad er tankerne med at to personer kan eksistere samtidig? To roller skal kunne pege på den samme person.

\textbf{Hvordan skal adgang for den finansielle afdeling være?}

Det ville være således at systemet pushede noget information. Det ville være nok med en rapport af en eller anden form.

\textbf{Danske \& udenlandske programmer} 

Udenlandske programmer kommer inklusiv rulletekster og kan derfor ikke gøres noget ved.

\textbf{Kunne det tænkes at fjerne rulletekster gav mindre promovering?}

Man kunne jo flytte logoerne til tidligere i rulleteksterne. Men i dag får man ingen promovering, fordi folk zapper videre med det samme.

Liste over brugte kategorier:

Han kan sende den.

Hvilke ricisi kan systemet have for TV 2?

Datakvalitet er meget vigtig, fordi der er nogen der skal have penge for det, så det er meget vigtigt at data er korrekt. Det er også meget vigtigt, hvad angår oppetid, man skal have et system der oppetidmæssigt kan måle sig med rulleteksterne (der altid virker).

Visning af programtider

Man kunne basere sig på EPG-backenden, der leverer data til TVTID.

Scope

Hvordan når man har consumer-enden, kunne man have et katalog form?

Man kunne forestille sig at man ville integrere det ind i TV 2 Play og TVTID. Jo mere brugervenligt en frontend er, jo oftere vil det blive brugt.

Brugervenlighed er et kunne hurtigt frem til det man søger.

Oplysninger om krediterede personer

Det nemmeste for TV 2 ville være at krediterede personer skulle kunne logge ind og ændre deres egne informationer.
